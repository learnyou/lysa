\newcommand{\ssa}{\subsubsection*}
\newcommand{\n}{\eta}
\newcommand{\Z}{\mathbb{Z}}

\newcommand{\1}{\hspace{1pt}}
\newcommand{\2}{\hspace{2pt}}
\newcommand{\3}{\hspace{3pt}}

\renewcommand{\,}{,\hspace{1pt}}
\renewcommand{\;}{;\hspace{3pt}}
\renewcommand{\~}{\backsim}

\renewcommand{\B}{\beta}
\renewcommand{\b}{\textbf}
\renewcommand{\i}{\emph}
\renewcommand{\l}{\lambda}
\renewcommand{\ng}{\mathrm{neg}\2}
\renewcommand{\v}{\curlyvee}

\chapter{Introduction}

This book is about Commutative Algebra - a fun area of math. That statement is a
bit redundant - is there an area of math that isn't fun? Of course not! Now
let's get started!

\begin{quotation}
  Everything is a toy if you play with it.

  -- Chris Pratt, Parks and Recreation
\end{quotation}


\section{So what is Commutative Algebra?}

Simply put, Commutative Algebra is the study of things called \i{commutative
  rings}. There are two words there, ``commutative,'' and ``ring.'' Before we
can get to those words, you have to understand some other words first.


\subsection{Sets and elements}

Imagine a bunch of things. These things all have some property in
common. Congratulations! You just intuited a \i{set}! See, you're getting this!

For example, look at \i{integers}. You may have heard of ``whole numbers,''
numbers that don't have a decimal point. Whole numbers are $\{0, 1, 2, 3,
\ldots\}$. Integers are the same, but they include negative numbers. The
integers are $\{\ldots, -5, -4, -3, -2, -1, 0, 1, 2, 3, 4, \ldots\}$

\subsubsection{Notation}

Did you see that I cleverly tricked you into understanding some notation there?
I'm also guessing you understood it! The notation I used is the brace notation
for sets, as well as set extrapolation. Those words are a bit confusing.

The ``brace notation'' simply means I use curly braces (these things - $\{$ and
$\}$) to express a set. Everything between those two braces is in the set. I
also used commas to differentiate between the various things in the set. See,
this math stuff isn't hard!

I also used this thing called ``assumed extrapolation.'' That was what you did
when you saw the ellipsis (the three dots - ``$\ldots$''). See, you're getting
this stuff without even thinking about it! The ellipsis basically means ``I
think you're smart enough to figure out what I'm going to put here.'' And you
are! Just look at how far you've gotten all on your own!

My Calculus teacher in high school was rather emphatic that mathematical
notation is all about being lazy. That idea stuck with me. I think you'll learn
very quickly is that mathematicians are lazy. Really lazy. We are so lazy that
we can't even bear to type the word ``integer.'' So, instead, we use letters to
express these words. We use a large fancy Z, instead of writing ``integers.'' We
really are that pretentious. So, whenever you see $\Z$, you can think ``oh, he's
just a lazy jackass who can't be bothered to type the word `integers'.'' And
you'll probably be right.

\subsubsection{Alright, back to the boring stuff}

Well, math isn't boring! I almost got you there!

So now we have a set of things. ``Thing'' is a bit of a generic term, and we
sometimes can get confused if we use it too much. So, when we talk about a thing
in a set, we usually use the word \i{element}. Don't be scared! You'll pick up
on the jargon eventually, and you'll be speaking like your incomprehensible math
professor!

Alright. Now, imagine that you can take an element in the set, do something to
it, and then get another element in the set? That's called a ``unary operator.''
``Unary'' means that it the operation needs one element from the set to do it's
job.

Take negation for example. Imagine the set of integers. Well, you can take $2$,
and apply the ``negation'' operation to it, and get $-2$, which is still an
integer. You can even take $-2$, and negate it, and get $2$ back.

When an operator takes any element from a set, and gives you back another
element from the set, we like to say that the set is \i{closed} under the
operation. In this example, $\Z$ is closed under negation. This is another
instance of mathematicians being lazy jackasses. We like to invent words to make
ourselves seem smarter, when, really, we're just like everyone else, minus the
social skills.

\subsection{Speaking of minus...}

You've probably heard the word ``minus'' before. Imagine you have five blocks,
and then you take two of them away. You now have three blocks. Congratulations,
you just \i{subtracted}! 

Again, us mathematicians are too lazy to say ``when you have five, and you take
away two, you get three.'' So, instead, we say ``five minus two equals three.''
But even that is too verbose. We reduce all those words into symbols, and write

\[ 5 - 2 = 3 \]

In this instance, $-$ and $=$ are called ``infix operators.'' That means they
are \b{fix}ed \b{in}between the two things on which they operate.

\subsubsection{Exercise}

Try subtracting some integers from each other, and confirm that $\Z$ is closed
under subtraction.

\subsubsection{Hey, guess what!}

You remember how $-$ and $=$ are infix operators? (I hope you remember, it was
only one paragraph ago!) Well, they operate on two things, and produce a third
thing. You know what these are called? \i{Binary operators}. See, you're getting
this!

\subsubsection{Notation}

I'm going to try to introduce some notation here. We won't use it very much in
the near future, but it will be useful later on. Don't worry if you don't
understand it yet, you'll undoubtedly pick it up as we move along.

Let's look at $-$ in the context of $\Z$. $-$ is a binary operator - it takes two
integers, and gives you another integer. Well, we're going to be pretentious,
and write that like 

\[ (-) : \Z \to \Z \to \Z \]

This is called ``Curry'' notation, named after a mathematician named Haskell
Curry. I write it in that flattened notation, because it's simpler. 

This Curry notation is ``left-associative.'' You don't need to know what that
means at the moment, intuitively. It really only matters when we're talking
about notation.  In the context of notation, that just means we should be
writing

\[ (-) : \Z \to (\Z \to \Z) \]

But again, mathematicians are lazy, and like to confuse people, so we remove the
parentheses where we don't absolutely need them.

You might ask, ``if we're removing parentheses that we don't absolutely need,
then why did you put parentheses around $-$?'' 

See, I knew you were smart! Asking all the right questions. I put the
parentheses around $(-)$ to specify that it's an infix operator. If it was used
as a \i{prefix} operator, I wouldn't use parentheses. ``Prefix operator'' just
means that the operator symbol is, by convention, placed before the things it
needs to do its job.

Let's look at the negation operator, for an example of a prefix operator. Most
people like to use $-$  as a negation operator. I find that rather confusing,
because it's also used for subtraction, which is a binary infix
operator. Mathematicians are stupid sometimes, too!

Instead, I'm going to write $\ng m$, to mean ``m, negated.'' $\ng$ is a prefix
operator. It's put $before$ the things that it uses to do its job.

We would write $\ng$'s \i{type signature} (the Curry thing I talked about
earlier) as 

\[ \ng : \Z \to \Z \]

You might be tempted to write 

\[ \ng 1 = -1 \]

but that doesn't solve the problem that we just tried to solve, because you're
still using $-$ for two separate things. Again, we're going to be lazy. $\ng 1$
is just too cumbersome to type. Instead, we're going to write it $` 1$
(pronounce it ``tick one'').  $\ng$ can now be written as

\[ \ng x = ` x \]

But guess what? We can be even lazier! It's implied that $`$ is a unary
operator. So, we will write

\[ \ng = ` \]

Mathematics is nothing more than sophisticated laziness.

When we write it that way, it seems as if $\ng$ is completely unnecessary. Well,
you're right! Guess what? I was tricking you again. I just tricked you into
learning the concept of $\n$-reduction! 

$\n$ is the Greek letter ``eta.'' It's pronounced like the word ``eight,'' but
with an ``uh'' sound at the end. 

$\n$-reduction just means that you write an equation with as few symbols as
possible. It's a bit more complicated than that, but you'll pick it up later!

\subsubsection{History lesson}

The ancient Greeks were too dumb to understand what eight is, so every time
someone brought it up, they would say ``uh'' immediately afterwords. The sound
``eight-uh'' became so common, that they decided to make it a letter. The
Greeks' poor comprehension of mathematics remains to this day, and is largely
the reason for their current financial crisis.

You will also notice that Greek letters are frequently used in physics. This is
the physicists' way of conceding that they actually have no idea what they are
talking about, and pleading for help from the mathematicians.

\subsection{Groups}

So, a group is actually pretty simple. When you have a set, and there is a
binary operation, you have a group. There are a few other things that need
to happen:

\begin{enumerate}
\item The set must be closed under the binary operation. We've discussed the set
  $\Z$, with the binary operation $-$. This property means that we have to be
  able to subtract an integer from another integer, and still get an integer.

  If you did the exercise in \S 1.1.3.1 (you \i{did} do it, right?), then you've
  already proven that $\Z$ is closed under $-$.
  
  If you are a rebel, and didn't do the exercise, do it now. This is what you
  get for putting things off!

\item The binary operation must be \i{associative}. That just means that you can
  group parentheses however you damn well please.  We have to be able to write

  \[ (a - b) - c = a - (b - c) \]

  Oh wait! That isn't true! So, $\Z$, together with $-$, is not a group!
  Instead, let's look at addition, the opposite of subtraction. We use $+$ as
  the infix operator for addition. 
  
  \[ (a + b) + c = a + (b + c) \]
  
  Phew! There we go, now let's move on.

\item There is some identity element for your binary operation. 

  \begin{itemize}
  \item Pick some random element in your set, and give it to your binary
    operator. 
  \item Remember, binary operators need two elements to do their job! You aren't
    trying to cheat, are you?
  \item Alright, the identity element is an element in your set, when you supply
    the binary operator with some random element, as well as the identity
    element, the operator spits out the same random element that you picked. 

    This identity element has to be the identity for the entire set. If we had
    to find a different identity element for every random element in our set, it
    would defeat the purpose of the identity element.
  \end{itemize}

  With the integers as your set, along with addition as your binary operator,
  the identity element is $0$. That is, $a + 0 = a$, and $0 + a = a$.
  
\item Every element in your set has an inverse element. That basically means

  \begin{itemize}
  \item Take a random element in your set.
  \item Give it to your binary operator.
  \item There needs to be another element in your set, that when you give it to
    the binary operator, the binary operator spits out your identity element.
  \end{itemize}
  
  Pretty cool, huh? 
  
  So, let's look at $\Z$ with $+$. Let's take some random element in $\Z$, and
  call it $x$. There needs to be some element, let's call it $y$, that $x + y =
  0$. Can you guess what it is?

  If you guessed $` x$, you're absolutely right! You're really picking this up
  quickly! So, $23 + `23 = 0$.

\end{enumerate}

So, we've just shown that $\Z$, along with $+$, is a group! That's pretty cool.

We're mathematicians (yes, you too, silly!), so we're going to be lazy. Instead
of writing ``$\Z$, along with $+$,'' we're going to write $(\Z, +)$.

In general, I will use a normal capital letter for the set, and a back~
(this character $\to \3 \~$) as the operation. When I talk about a group $(A,
\~)$, I'm talking about a set $A$, with the operation $\~$.

It seems to be the convention in other texts to use $0$ as the identity element
for Abelian groups, even when your set has nothing to do with numbers. I don't
like this. So, instead, I'm going to write $I_\~$. In the long run, this is far
less confusing than using $0$ wrecklessly.

\subsubsection{More notation (great...)}

Remember when I said that $23 + `23 = 0$? Well, adding a negative integer is the
same as subtracting an integer! In the interest of being lazy, we can write $23
- 23 = 0$.

\subsubsection{A small note}

I think that the unary $-$ notation was invented because it was convenient to
think of negative numbers as zero minus whatever your number is. 

When we're dealing with rings, the notion of ``zero minus whatever'' becomes
really confusing and cumbersome. Therefore, throughout the rest of the book, I
will disregard subtraction entirely, and use the backtick notation whenever I
need to express the concept.

So, to summarize, disregard the previous section.

\subsection{Abelian groups}

An Abelian group is just like a normal group, except your binary operation must
be \i{commutative}.

Hey! We just found the first word we set out to define!

Alright, that paragraph is over with, calm down.

``Commutative'' just means that it doesn't matter in what order you give things
to your group's binary operator.

So, for $(\Z, +)$, it's obvious that $a + b = b + a$. So, integers are an
Abelian group!

The idea of abstract algebra is to think beyond our preconceived notions. So,
I'm going to be mean, and make you think of some arbitrary group $(A,
\~)$. With $(A, \~)$, it must be true that $a \~ b = b
\~ a$. See, I told you math is easy!

These Abelian groups are also called ``commutative groups.''

\subsection{Rings}

\begin{enumerate}
\item So, take your Abelian group.
\item Now, add another binary operation, under which your group is closed.
\item That's it. What, did you think math was complicated? What have I been
  telling you this whole time? Jeez.
\end{enumerate}

So, do that, and you get a ring. As usual, you have to wade through some more
horseshit to make sure that you actually have a ring, and not some weird
perverted topology thing.

In addition to all of the rules with the Abelian groups, you have to satisfy
some more rules. Jeez, it's like dealing with your parents after they found your
pot stash. Rules, rules, rules.

\begin{enumerate}
\item The second binary operation has to be associative. I already explained
  what that word means. I'm a mathematician, and I'm lazy, so I'm not going to
  explain it again.

\item There is an identity element for your second binary operator. 

  Let's look at $(\Z, +)$. Let's pick multiplication as our second operator. You
  remember multiplication, right? We use $\times$ as the symbol for
  multiplication.

  \[ 4 \times 3 = 4 + 4 + 4 \]
  \[ 3 \times 4 = 3 + 3 + 3 + 3 \]
  
  Basically, you just add the first number to itself a number of times, given by
  the second number.
  
  Let's look at $(\Z, +)$, with $\times$ as the second binary operation. With
  $(\Z, +)$ and $\times$, $1$ is the identity element.
  
  Now, let's try to think more abstractly. Let's consider some arbitrary Abelian
  group. $(A, \~)$. Let's assume there's another binary operator on $A$, and
  we'll call it $\v$. This just says that there needs to be another identity
  element for $\v$, and we'll call it $I_\v$. I told you this isn't hard!

  With regards to $(A, \~)$, along with $\v$, we need to be able to pick some
  element from $A$, let's call it $a$. This has to be true:

  \[ a \v I_\v = a\]
  \[ I_\v \v a = a\]
  
\item Your second binary operator must \i{distribute} over the first one.
  
  With $(A, \~)$, along with $\v$, this must be true

  \[a \v (b \~ c) = (a \v b) \~ (a \v c)\]
  \[(b \~ c) \v a = (b \v a) \~ (c \v a)\]
  
  where $a$, $b$, and $c$ are elements of $A$.
  
\end{enumerate}

\subsubsection{Notation}

\begin{itemize}
\item I'm tired of writing ``the ring $(A, \~)$, along with the second binary
  operation $\v$.'' It's also 5:30 AM. So, instead, I'm going to write $(A, \~,
  \v)$.
  
\item Writing ``$a$, $b$, and $c$ are elements of $A$'' is proving to be
  cumbersome. Instead, I'll use the remarkably standard notation $a, b, c \in A$.
\end{itemize}

I'm trying to gently introduce you to the world of notation. How's it working?

\subsubsection{Exercise}

In the last property, I didn't give any examples of distributivity in the ring
$\Z$. Write the analog of the last property with $(\Z, +, \times)$.

\subsection{Commutative rings}

A careful reader would note that I didn't specify that $\v$ had to be
commutative. That is, in the context of $(A, \~, \v)$, for $a, b \in A$, it
isn't necessary that $a \v b = b \v a$.

The only time this needs to be true is if either $a$ or $b$ is $I_\v$. 

\b{If} $a \v b = b \v a$, for all $a, b \in A$, then you have a \i{commutative
  ring}! If not, then $(A, \~, \v)$ is an ordinary ring.

Yay! After 500 lines of \LaTeX, we finally got to the point!

\subsubsection{Notation}

Since this is a book about commutative rings, it wouldn't make sense to have to
specify that each ring that we're dealing with is commutative, would it? So,
from here on out, assume that all rings are commutative, unless I tell you
otherwise.

On the topic, it's a bit cumbersome to have to write $(A, \~, \v)$ when I want
to discuss an ordinary commutative ring. So, instead, I'm going to rely on you
to assume that any set $A$ that I talk about from now on is a commutative
ring. 

If I tell you that $A$ is not a commutative ring, please do not take the
previous paragraph to assume that I'm lying, and that $A$ really is a
commutative ring.
\section{Am I smart enough to read this?}

\begin{quotation}
  Everybody is a genius. But if you judge a fish by its ability to climb a tree,
  it will live its whole life believing that it is stupid.
  
  -- Albert Einstein
\end{quotation}

This book is intended for anyone who has an interest in math. You don't have to
have a PhD to understand this stuff. It's actually pretty simple.

As I point out many times later on, mathematicians are really lazy. We like to
make ourselves sound smarter by expressing simple ideas in incomprehensible
notation.

I guarantee, by the end of reading this, you too can act like a pretentious
jackass, and start writing things like

$$\forall a \in R,\, \exists (-a) \in R \mid a + (-a) = 0$$

Albert Einstein is full of great quotes. Another one of my favorites is ``If you
can't explain it to a six year old, you don't understand it yourself.'' That's
the mindset I took when writing this. How can I express these simple concepts
simply?

This book is (hopefully) written so that a six year old could understand
it. That said, if there are any six year olds reading this, please ignore the
cursing.

It would definitely help if you have elementary school mathematics under your
belt. Even that is not entirely required. As long as you can employ logic, and
can read English fluently, you shouldn't have any trouble reading this!
