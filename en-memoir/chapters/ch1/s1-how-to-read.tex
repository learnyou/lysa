\s{How to read the book}

The best way to read this book is to just read it. Don't skip sections,
or look ahead, or anything like that. Just read it straight
through. It's also pretty important that you read the rest of this
chapter. I promise it's not too boring.

Do all of the exercises. There aren't that many. However, they are
pretty difficult. The exercises all have solutions, which are in
\cref{ex-solutions}.

The exercises are designed to make you think, and widen your perspective
on the topic at hand. They are not designed to be tedious. They are
difficult, but the good kind of difficult.

It would be perfectly okay to just do the exercises (all of them), and
then go back and read the text when you don't understand something.

\Cref{appendix-d} is a reference section. It contains every single
theorem, definition, identity, and property in this book. Unlike the
contents of this book, \cref{appendix-d} is not meant to be read
straight through. However, if you don't remember the name of something,
or want to know if some property is true, \cref{appendix-d} is the place
to look.

Well, I lied. It would be much faster to just Google your problem. If Google
can't find it, use \cref{appendix-d} as a last resort.