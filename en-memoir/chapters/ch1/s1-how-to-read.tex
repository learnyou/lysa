\s{How to read the book}

The best way to read this book is to just read it. Don't skip sections, or look
ahead, or anything like that. Just read it straight through. It's also pretty
important that you read the rest of this chapter. I promise it's not too boring.

Do all of the exercises. There aren't that many. However, they are pretty
difficult. The exercises all have solutions, which are in \cref{ex-solutions}.

The exercises are designed to make you think, and widen your perspective on the
topic at hand. They are not designed to be tedious. They are difficult, but the
good kind of difficult.

It would be perfectly okay to just do the exercises (all of them), and then go
back and read the text when you don't understand something.

There are two automatically generated versions of the book.

\begin{itemize}
  \item {\rmfamily The version optimized for printing uses this font for text,
      and the other font for headers.} If you are reading the print version:
    yes, I understand the margins are ridiculous; it's for readability, though.
  \item {\sffamily The version optimized for reading on an electronic screen
      uses this font for text, and the other font for headers.}
\end{itemize}

I personally prefer to print things out and read them by hand, I find it to be
much more conducive to learning, mostly because I can also make notes in the
margins and what not. In this book, I tried to make the explanations of things
clear enough that you won't \xti{have} to write notes in the margins. I probably
failed to do so at least once or twice. If you don't like something in the book,
you can \emailme or complain in the \bugtracker .
