\s{Conventions used throughout}

You don't actually have to read this section, but it would be useful.

\begin{enumerate}
  \item The \S\ symbol refers to a section. So \S\ 3.2 means ``chapter 3,
    section 2''.
  \item Even though most of the writers are American, I still use the British
    convention of putting periods after quotation marks: ``like this''. The
    British convention is less ambiguous. If you see the American convention
    anywhere in this book, please report it in the \bugtracker\ or \emailme .
  \item ``I'' refers to me. ``We'' refers to both me and you, the
    reader. ``You'' refers to, you guessed it, the reader. It's the convention
    in academia to use the so-called ``royal we'', such as ``we subtract 2 from
    both sides of the equation to obtain the result \dots''.

    Sometimes, we will accidentally use the royal we, out of habit. Crap, I just
    did it there! See? It's very difficult to avoid. Like any of the other
    conventions herein, if you see it broken, please report the error to the
    authors. You can use the \bugtracker, or, if you don't want to make a gratis
    GitHub account, you can \emailme.

  \item Oh yeah, sometimes I'll use \code{monospace} in things like URLs or
    emails for the sake of disambiguity. If it's an email or a URL, I'll
    surround it with angle-brackets (these things: \code{<>}) so you don't
    confuse it with the surrounding text.

  \item If you see some number as a superscript in the middle of text: like
    this\footnote{Hey, you found me!}, then the number refers to a footnote. If
    the superscript number is in the middle of math, it's probably just math.

  \item If there's some number in brackets, like this: \cite{lyah}, then it's a
    citation. If you're reading this as a PDF, you can actually click on the
    number, and your PDF reader will take you to the relevant bibliography
    entry. Go ahead, check it out! I'll wait. You can do the same thing for
    footnotes and URLs.\footnote{Well, clicking the URL will open up your web
      browser, but you get the point}

  \item All of the equations in this book are numbered (at least they should be)

    \begin{equation}
        \label{eq:example}
        f(x) = 2 + x 
    \end{equation}

    The idea is, if you notice an error, you can tell us ``there is an error in
    equation 2.21''. Although, most people will just use the page number, which
    is easier for everyone.

    That said, you are much more likely to notice somewhere where I forgot to
    number an equation, in which case just tell me the page number.

    \label{conventions}
  \item If you see something \term{in italics}, it's usually a vocabulary word.
\end{enumerate}