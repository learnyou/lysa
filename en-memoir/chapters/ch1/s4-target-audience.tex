\s{Target audience}

The target audience is people who want to learn more about math. More
specifically, it's for people who want to learn how to use mathematics as a tool
to abstract their thoughts; to use it as a language to express their ideas. Math
is a very powerful and very fun tool to do that.

Math happens to be incredibly useful, and that's just lovely. But most
mathematicians don't care. The math is interesting and fun (no, really!), and
that's why people study it.

When I was first writing the book, I wrote it in an effort to strengthen
my own understanding. So, the target audience was me. The very first
versions of this book were about a abstractish branch of math called
commutative algebra. Later on, it seemed more fitting to abstractly go
over the basics of math. That's what the current version of the book
does.

The prerequisites are that you know all of the basics about arithmetic: you
should know how to add, subtract, multiply, divide, and exponentiate. You should
also be familiar with the basic concept of variables, and maybe even how to
solve systems of equations.
