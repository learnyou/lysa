\ch{Props, proofs, and providence}
\label{ch:props}
\label{props}
\label{ch:logic}
\label{logic}

Alright, you survived the first chapter! (Or, I guess, the second
chapter).

This chapter goes over intuitionist logic, particularly that which relates to
type theory. If you are already familiar with inituitionist logic and with type
theory, you can skip to the next chapter. This chapter contains a lot of
nitty-gritty logical proofs that might not be fun.

It's important that you understand the logical foundations of math before we
move on to actual math. That's why this chapter exists.

Anyway, let's get this over with.

For a long time, mathematicians would deal with statements that were either
true or false. This was called \term{classical logic}. 

In the 1930s, a man named Kurt \godel\ proved a rather shocking fact: in any
non-trivial formal arithmetic system, there will always be statements that are
true, but cannot be proven. So, if we can't \xtb{prove} something, how do we
know whether or not it's true?

\godel\ was somewhat of a character. He was extremely paranoid about being
poisoned, so he refused to eat anything unless his wife prepared it for him.
Towards the end of his life, his wife was hospitalized for an extensive period
of time, during which \godel\ refused to eat, and therefore starved to
death. \cite{w-godel}

Getting back on the topic, with regard to how to decide whether or not
unprovable things are true, the answer is to not think of things in terms of
\term{true} and \term{false}, but rather in the terms of \term{provable} and
\term{unprovable}. Anything that's provable is true. Anything that isn't
provable, we don't care whether or not it's true, because we can't prove it.
\nocite{w-godel-incompleteness}

Further supporting this line of thinking is the liar's paradox:

\begin{quote}
    This statement is false.
\end{quote}

If we assume that the statement is true, then the statement makes itself
false. If you assume the statement is false, it makes itself true. This is an
example of a statement that cannot be said to be true or false. There are
similar statements in math, were we can't even say it's true or false.

In the case of the liar's paradox, we have a pretty quick cop-out, by just
saying that it's not provable. We don't care whether or not it's true.

So, how do you express the notion that you have a proof something? Let's call
the proof $p$ and the proposition $X$. To say ``$p$ is a proof of $X$'', you
write:

\begin{equation}
    p : X
\end{equation}

\xtb{Wait, that's the type notation!}

Yep! This is what's called the \term{Curry-Howard correspondence}.\footnote{The
  ``correspondence'' part just means that the concepts correspond to each other,
  and that a guy named Curry and a guy named Howard discovered this around the
  same time. This \xtb{was not} a correspondence between Curry and Howard.}
Namely,

\begin{itemize}
  \item functions and proofs have identical semantics, and
  \item types and propositions have identical semantics
\end{itemize}

Cool! You already know how to send values to functions, and to manipulate
functions a bit. So, proofs should be easy! Let's get started then!

\input{chapters/ch3/s1-if-then.ltx}
\input{chapters/ch3/s2-junctions.ltx}
\input{chapters/ch3/s3-diffs.ltx}
\input{chapters/ch3/s4-diffs-continued.ltx}
\input{chapters/ch3/s5-diffs-continued-continued.ltx}
