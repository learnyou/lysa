\chapter{Sets, Relations, and Functions}
\label{ch:functions}
\answergraph{images/lambda-angry-gimped.png}

The first question addressed in a math class is often ``what is math?''. There
are a number of correct answers to this, but the most common is ``the study of
patterns''. I personally like that definition. It's useful to distinguish purely
mathematical notions of patterns from boring numerical calculations. That's what
abstract algebra is all about, and what this book is about: showing the cool
patterns that show up in math. It might be useful to show how said patterns
apply to things in the real world, but said applications are usually far less
interesting than the pattern itself.

Before I can show you what the patterns are, the two of us have to develop some
tools for expressing those patterns. Well, actually, the tools were developed a
long time ago, I'm just showing you. You know what, never mind, let's just get
started.

\section{Sets}

The first tool is the \xti{set}. A set is a collection of things. The things in
a set are called ``elements'' of the set. Sets do not have any intrinsic order,
which means that if you write down every element of a set, the order in which
you write the elements down doesn't matter. This means, for instance, that

\begin{equation}
  \mset{1,2,3,4} = \mset{2,4,3,1}
\end{equation}

Oh, yeah, I should probably mention that sets with only a few elements are often
denoted by just listing the elements between curly braces, separated by
commas. In addition, sets don't have any notion of multiplicity, meaning that
the number of times an element appears is irrelevant. This means

\begin{equation}
  \mset{1,2,3,4} = \mset{2,4,3,1,2,1,3,3,4,1,2}
\end{equation}

This means, equivalence on sets is defined in a much simpler way. That is, two
sets $A$ and $B$ are equal if

\begin{alignmath}{l}
  \fall x \in A, x \in B; \text{ and},\\
  \fall x \in B, x \in A.
\end{alignmath}

There are two basic tools in abstract math. These tools are general enough to
allow you to express almost any idea. Remember, the goal of this book is to
teach you how to use math as a tool for thinking. The tools I mentioned are
\term{functions} and \term{types}.

These two tools\dots\ really, this whole chapter isn't very interesting. If you
already know what functions and types are, what Currying is, and how to read a
set comprehension, I suggest that you skip to the next chapter.

I do, however, recommend that you do the exercises. They aren't very tedious. I
made a point throughout this book of writing non-tedious, but nonetheless
challenging exercises.

\s{Types}

Types are easier to understand, so I'll start with them. Every mathematical
object has a type. $3$, for instance, is a number. In math, it's useful to think
``what type of thing is this?''. That's all a type is.

Mathematicians love to be lazy. Instead of writing out ``$x$ is of type $T$''
every time, they'll instead write:

\begin{equation}
    x : T
\end{equation}

So, $3$ is a number. How do you express this?

This is actually a somewhat tricky question, just because there are a whole mess
of different types of numbers.

\begin{itemize}
  \item There's ``natural numbers'', which are
    $\mset{0, 1, 2, 3, 4, 5, \dots}$.\footnote{The first construction of the
      axioms of natural numbers had $1$ as the least natural number. It turns
      out that the theory is considerably easier and more interesting if $0$ is
      the least natural number. Nonetheless, some books still use $1$ as the
      least natural number. The guy who came up with the first construction was
      named Peano, so a lot of people, including myself, refer to the natural
      numbers starting with 1 as the ``Peano numbers''.}
  \item There's the ``integers'', which are
    \(\mset {\dots, -2, -1, 0, 1, 2, \dots}\).
  \item Then, there's the ``real numbers'', which I'll get to later on.
  \item There's ``imaginary numbers'', which are the opposite of real
    numbers.
\end{itemize}

The list goes on and on. I'll explain what all of these number types are later.
For now, we'll stick to the simplest type of numbers: natural numbers. The
symbol for natural numbers is $\N$.

\begin{equation}
    3 : \N
\end{equation}

Types even have a type. The type is called $\Type$.

\section{Functions}

Okay, so what's a function?

A function is a mathematical construct that takes some input, does something to
it, and sends back output. Essentially, instead of being some static value that
isn't very interesting, a function takes something as input, does something to
it, and spits back some output.

Let's make a function that takes some number, and adds 3 to it.

\begin{equation}
    \fn{x}{x + 3}
\end{equation}

\xtb{What is that notation?}

The first character is the Greek letter lambda: $\lambda$. It serves the same
function in Greek as the 'L' does in English. The $\lambda$ serves to indicate
that we are creating a function, not just an ordinary value.\footnote{Although a
  lot of people will use $\lambda$ as an ordinary variable name, which gets very
  confusing. Thankfully, I haven't seen anyone use \xti{both} the $\lambda$ for
  functions and for a variable name.}

Immediately following the $\lambda$ is a list of inputs to the function. In this
case, there is only one, which is called $x$. If there was more than one input,
they would be separated by commas. After the list of variables, there's a
weirdly shaped arrow: $\mapsto$. It should be read as ``maps to''.

The portion after the arrow indicates what the function does to the inputs. This
is called, uninterestingly enough, the \term{output} of the function.

That's all a function is: input and output. You've likely forgotten what the
function is, so I'll write it again:

\begin{equation}
    \fn{x}{x + 3}
\end{equation}

This function takes some number $x$ as input, and the output is $x + 3$. This is
not a very interesting function, I know.

Let's give this function a name

\begin{equation}
    f = \fn{x}{x + 3}
\end{equation}

The much more common notation is this:

\begin{equation}
    \eva{f}{x} = x + 3
\end{equation}

That is to be read ``$f$ of $x$ equals $x$ plus $3$''.

What's the type of $f$?

\begin{alignedmath}
    f : \N \to \N \\
    \eva{f}{x} = x + 3
\end{alignedmath}

Remember that $\N = \mset{0,1,2,3,4,5,6,\dots}$.

I think the $\N \to \N$ part is pretty self-explanatory: $f$ takes something of
type $\N$, and spits back out something of type $\N$.

So, what happens when we feed $80$ to $f$?

\begin{rclmath}
    \eva{f}{80} & = & 80 + 3 \\
                & = & 83 \\
\end{rclmath}

Yay! This math stuff isn't hard!

\ss{Comprehensions}

Think about this for a second: $f$ takes a number and adds 3 to it. What's the
lowest value in $\N$? Well, it's $0$. If you take $0$, and add $3$ to it, you
get $3$. You can't add $3$ to a natural number and get something less than
$3$.

You would therefore be more correct in saying that $f$'s output type is the set
of numbers greater than (or equal to) $3$.

\begin{rclmath}
    f & : & \N \to \comprehension{z : \N}{z \ge 3} \\
    \eva{f}{x} & = & x + 3 \\
\end{rclmath}

\xtb{What is that curly brace nonsense?}

It's called a \term{comprehension}.

\answergraph{images/comprehensions.png}

Read it in two parts: before the pipe, and after the pipe.

\begin{itemize}
  \item Before the pipe describes what every item of the type looks like. In
    this case, it's just a number $z$.

  \item After the pipe has a list of conditions that must hold true. In this
    case, $z$ must be greater than or equal to $3$.
\end{itemize}

\answergraph{images/comprehensions3.png}

Let's look through some more comprehensions.

\begin{example}
    The type of all natural numbers greater than $60$, but less than $70$:

    \begin{equation}
        \comprehension{x : \N}{60 < x < 70}
    \end{equation}

    Alternatively, you can use the logical-and operator:

    \begin{equation}
        \comprehension{x : \N}{\parens{60 < x} \text{ and } \parens{x < 70}}
    \end{equation}

    That should be read as ``the type of all numbers $x$ such that $60$ is less
    than $x$ and $x$ is less than $70$''.
\end{example}

\begin{example}
    A function that takes some number $x$, and constructs the type of all
    natural numbers less than $x$

    \begin{equation}
        \fn{x}{\comprehension{y : \N}{y < x}}
    \end{equation}
\end{example}

\begin{example}
    The type of all natural numbers that are a multiple of two:

    \begin{equation}
        \comprehension{2x : \N}{x : \N}
    \end{equation}

    This is a bit odd. In this case, it's easier to read the stuff after the
    pipe first:

    \begin{equation}
        x : \N
    \end{equation}

    So, for every number $x$, we're going to multiply it by $2$. Therefore,
    there won't be any numbers in the result that aren't multiples of $2$. Yay!
\end{example}

\begin{example}
   The Fibonacci numbers are the numbers 

   \begin{equation}
       F = \mvec{1,1,2,3,5,8,13,21,34,55,89, \dots}
   \end{equation}

   \xtb{Why did you use round parentheses instead of curly braces?}

   The Fibonacci numbers have repetition, and the order in which you list them
   does matter. To represent this, you use round parentheses when listing them
   instead of curly braces.\footnote{The mathematical difference is that the
     Fibonacci numbers are a \term{list}, while the other types we've
     encountered are all \term{sets}. I'll get to these two in depth later on in
     the book.}

   The $n$th Fibonacci number is the last two Fibonacci numbers added
   together. $89 = 55 + 34$. $55 = 34 + 21$. Et cetera.

   This is a function that gives the $n$th Fibonacci number:

   \begin{rclmath}
       \eva{f}{0} &=& 1 \\
       \eva{f}{1} &=& 1 \\
       \eva{f}{n} &=& \eva{f}{n - 1} + \eva{f}{n - 2}
   \end{rclmath}

   When indexing items, you always start at $0$. It's just a thing.

   This is an example of \term{recursion}: the definition of $f$ uses itself in
   its definition.
\end{example}

\begin{example}
    Natural numbers greater than $1$ can be divided into two categories:

    \begin{description}
      \item[Composite numbers] can be divided by another natural number to get a
        third natural number. For instance, $15$ can be divided by $3$ to get
        $5$.

        There are always two \term{trivial divisors}. $1$ can trivially divide
        every number. Likewise, every number can trivially divide itself. These
        trivial divisors occur in every single number and thus are not very
        interesting.

      \item[Prime numbers] have only trivial divisors. For instance, there is no
        non-trivial divisor of $7$.
    \end{description}

    \xtb{Why do you exclude $1$?}

    Well, because $1$ has only trivial divisors.

    \xtb{Okay, but what about $2$?}

    Well\dots hell, I dunno\dots

    Oh, oh, I remember! There's this cool thing called the Sieve of
    Eratosthenes, I'll show it to you in a little while.\footnote{TODO: show
      this to reader.} Anyway, the sieve makes sense if $2$ is prime but $1$
    isn't.

    Let's construct the type of composite numbers:

    \begin{equation}
        \label{eq:composite}
        \scomp{x \cdot y : \N}{
          \parens{x : \N}
          \text{ and }
          \parens{y : \N}
          \text{ and }
          \parens{x > 1}
          \text{ and }
          \parens{y > 1}
        }
    \end{equation}

    Let's think about this:

    If you hold $x = 2$ (the minimum value of $x$), let's see what the output
    values are:

    \begin{eqnarray}{cl}
       & \scomp{2 \cdot y : \N}{
           \parens{2 : \N}
           \text{ and }
           \parens{y : \N}
           \text{ and }
           \parens{2 > 1}
           \text{ and }
           \parens{y > 1}
         } \\
       = & \scomp{2 \cdot y : \N}{
             \parens{y : \N}
             \text{ and }
             \parens{y > 1}
           } \\
       = & \mset{2 \cdot 2, 2 \cdot 3, 2 \cdot 4, 2 \cdot 5, 2 \cdot 6, \dots} \\
       = & \mset{4, 6, 8, 10, 12,  \dots} \\
    \end{eqnarray}

    That is, all of the multiples of $2$ greater than $2$. Again, $2$ trivially
    divides $2$, so I excluded it (this is why $x$ has to be greater than $1$,
    because otherwise $2$ would be in the list of multiples of $2$).

    Now, let's hold $x = 3$

    \begin{equation}
        \mset{6, 9, 12, 15, 18, 21, 24, \dots}
    \end{equation}

    With $x = 4$

    \begin{equation}
        \mset{4, 8, 12, 16, 20, 24, 28, \dots}
    \end{equation}

    You get the point. If you follow through with this (infinitely), you would
    eventually list every single composite number.

    \xtb{What about $2$? Isn't $2$ a composite number?}

    Nope! $2$ can only be divided by $1$ or $2$, which makes it prime.

    \xtb{What about dividing by $0$?}

    You aren't allowed to divide by $0$.

    \xtb{Why}?

    Because allowing people to divide by $0$ would poke holes into our logic.

    \xtb{But if you take a cookie, and break it into $0$ parts, then you get
      $0$, right?}

    Jesus, with the questions\dots Will you give it a rest? I'll explain later,
    I promise.
\end{example}

I'm going to give you some exercises. Please do them. They'll really help you
with the material.

\begin{ExerciseList}
    \Exercise{Using that weird type bracket notation, construct the type of all
      numbers greater than $80$, and less than $200$.}
    \Answer{\(\comprehension{x : \N}{80 < x < 200 }\)}

    \Exercise{Construct the type of all numbers less than or equal to $30$, and
      greater than $10$.}
    \Answer{\(\comprehension{x : \N}{10 < x \le 30 }\)}

    \Exercise{
      \label{xstotype}

      Construct a function that takes two numbers $x$ and $y$, and uses
      them to construct the type of numbers greater than $x$ but less than $y$.

      Hint: to have multiple values sent to a function, just add commas: $\fn{x,
      y}{x + y}$}
    \Answer{\(\fn{x, y}{\comprehension{a : \N}{x < a < y }}\)}

    \Exercise{Figure out the type of the function you constructed in
      \cref{xstotype}.}
    \Answer{$\mlist{\N, \N} \to \Type$}

    \Exercise{The factorial of a number $n$, written

      \begin{equation}
          n!
      \end{equation}

      Is $n$ multiplied by all of the numbers less than $n$. Write a recursive
      function to get the factorial of any natural number.

      Hint: $0! = 1$.
    }
    \Answer{
      Here's what I went with:

      \begin{eqnarray}
          0! = 1 \\
          n! = n \cdot \parens{\parens{n - 1}!}
      \end{eqnarray}
    }
\end{ExerciseList}
\section{Currying and Uncurrying!}

Those exercises were fun, weren't they? I bet not. If you had a lot of trouble
with them, don't worry. They're purposefully difficult.

My experience is that most textbooks have a ton of pedantic, tedious
exercises. I try to not do that. The exercises I gave you are difficult, but the
good kind of difficult.

Anyway, let's move on, hopefully to something a little bit less boring.

Remember $\Z$, the integers? If you don't:

\begin{equation}
    \Z = \mset{\dots,-5,-4,-3,-2,-1,0,1,2,3,4,5,\dots} \\
\end{equation}

Consider this function:

\begin{eqnarray}
    f : \parens{\Z, \Z, \Z} \to \Z \\
    \eva{f}{a,b,c} = a + b + c
\end{eqnarray}

What's $\eva{f}{1,2,4}$? Well,

\begin{eqnarray}
    \eva{f}{1,2,4} & = & 1 + 2 + 4 \\
                   & = & 7 \\
\end{eqnarray}

What happens when you only apply one value to $f$?

\begin{eqnarray}
    \eva{f}{18,b,c} & = & 18 + b + c \\
\end{eqnarray}

I've sort of transformed $f$: instead of being a function that takes three
integers that adds them together, it's now a function that takes one integer,
and returns back a function that takes two integers, and adds them to the first
integer.

That word jumble is confusing. Let me make it symbolic for you:

Previously, $f$ was like this:

\begin{equation}
    f : \parens{\Z, \Z, \Z} \to \Z
\end{equation}

However, if you allow only one value to be sent to $f$, then you can think of
$f$ more like this:

\begin{equation}
    f : \Z \to \parens{\parens{\Z, \Z} \to \Z}
\end{equation}

Let's rewrite $f$ with this new definition in mind:

\begin{equation}
    \eva{f}{a} = \fn{y,z}{a + y + z}
\end{equation}

This might be a little clearer if I used the non-named function notation.

\begin{equation}
    f = \fn{a}{\parens{\fn{y,z}{a + y + z}}}
\end{equation}

Well, let's just consider that second inside function.

\begin{equation}
    \fn{y,z}{a + y + z}
\end{equation}

(This assumes that $a$ was already defined elsewhere, which it would have been
if we put it back where it goes).

Let's look at the type of the inside function

\begin{equation}
    \fn{y,z}{a + y + z} : \mvec{\Z,\Z} \to \Z
\end{equation}

What if you only send the first value to the inside function?

\begin{equation}
    \fn{14,z}{a + 14 + z} : \Z \to \Z
\end{equation}

Well, again, we just transformed the inside function from

\begin{equation}
   \mvec{\Z,\Z} \to \Z
\end{equation}

to something like

\begin{equation}
   \Z \to \parens{\Z \to \Z}
\end{equation}

Well, now, let's sort of consider the original function, $f$:

$f$'s type is now

\begin{equation}
   f : \Z \to \parens{\Z \to \parens{\Z \to \Z}}
\end{equation}

And $f$ is more properly written like this:

\begin{equation}
    f = \fn{a}{\parens{\fn{b}{\parens{\fn{c}{a + b + c}}}}}
\end{equation}

This is called \term{Currying}.

\begin{definition}[Currying]
    \label{def:currying}
    Taking something with type

    \begin{equation}
        \mvec{a,b,c,d,e,\dots} \to \beta
    \end{equation}

    and turning it into something with the type

    \begin{equation}
        a \to \parens{b \to \parens{c \to \dots}}
    \end{equation}
\end{definition}

\term{Uncurrying} is doing exactly the opposite thing

\begin{definition}[Uncurrying]
    \label{def:uncurrying}
    Taking something with type

    \begin{equation}
        a \to \parens{b \to \parens{c \to \dots}}
    \end{equation}

    and turning it into something with the type

    \begin{equation}
        \mvec{a,b,c,d,e,\dots} \to \beta
    \end{equation}
\end{definition}

I'll give some examples of Currying and Uncurrying here, as well as some
exercises. After that, though, I'm going to silently curry \& uncurry types
without explicitly saying that I'm doing so. I do it so often that it's too much
effort to say so every time. You'll get used to it, I promise.

One more thing. I don't like typing out all of the parentheses in this:

\begin{equation}
    a \to \parens{b \to \parens{c \to \dots}}
\end{equation}

Especially when most of the time, they group to the right like that. Instead,
I'm going to write:

\begin{equation}
    a \to b \to c \to \dots
\end{equation}

\ss{Examples}

\begin{example}
    The curried form of:

    \begin{equation}
        \mvec{\Z,\N} \to \Q
    \end{equation}

    is:

    \begin{equation}
        \Z \to \N \to \Q
    \end{equation}
\end{example}

\begin{example}
    The curried form of

    \begin{equation}
        \mvec{a, a \to b} \to b
    \end{equation}

    is:

    \begin{equation}
        a \to \parens{a \to b} \to b
    \end{equation}

    A function that does this is:

    \begin{eqnarray}
        f : a \to \parens{a \to b} \to b \\
        \eva{f}{x,g} = \eva{g}{x}
    \end{eqnarray}
\end{example}


\begin{example}
    \label{exp:flip}
    The curried form of

    \begin{equation}
        \parens{\mvec{a, b} \to c} \to \parens{\mvec{b,a} \to c}
    \end{equation}

    is:

    \begin{equation}
        \parens{a \to b \to c} \to b \to a \to c
    \end{equation}

    The function with this exact type is $\flip$.

    \begin{eqnarray}
        \flip & : & \parens{a \to b \to c} \to b \to a \to c \\
        \eva{\flip}{f,x,y} & = & \eva{f}{y,x} \\
    \end{eqnarray}
\end{example}

% TODO: Add more exercises
\begin{ExcList}
    \Exercise{Write out the uncurried and curried type of the $+$ function over
      natural numbers.}
    \Answer{Curried:

        \begin{equation}
            \N \to \N \to \N
        \end{equation}

        Uncurried:

        \begin{equation}
            \mvec{\N, \N} \to \N
        \end{equation}
    }

    \Exercise{There is a very important function, called ``compose''. The symbol
        for compose is a circle: $\circ$. Here is compose's type:

        \begin{equation}
            \circ : \parens{b \to c} \to \parens{a \to b} \to a \to c
        \end{equation}

        It is usually used in infix form:

        \begin{eqnarray}
            f : a \to b \\
            g : b \to c \\
            g \circ f : a \to c \\
        \end{eqnarray}

        Write the $\circ$ function. 
    }
    \Answer{$\eva{f \circ g}{x} = \eva{f}{\eva{g}{x}}$}
\end{ExcList}
\s{Partial functions}

If you remember way back to the beginning of the chapter, I constructed a
function that looked like this:

\begin{equation}
    \eva{f}{x} = x + 3
\end{equation}

I pointed out that, because there is no natural number $x$, such that
$\eva{f}{x} < 3$, that $f$'s output type would therefore be the set of numbers
greater than or equal to 3. That is

\begin{equation}
    f : \N \to \scomp{x : \N}{x \ge 3}
\end{equation}

In this case, we made $f$ into a \term{surjective} function, merely by making
its output type more specific.

Before I get any further, let me give you some definitions. You don't have to
remember all of these right away, you'll pick them up as we go along.

\begin{definition}[Domain]
    The domain of a function $f$ is the input type of $f$. That is, if

    \begin{equation}
        f : A \to B
    \end{equation}

    then the domain of $f$, written $\dom{f}$, is $A$. That is:

    \begin{equation}
        \dom{f} = A
    \end{equation}
\end{definition}

\begin{definition}[Total function]
    If a function $f$ is able to operate on every value in its domain, $f$ is a
    \xti{total function}.

    The function above, $\fn{x}{x + 3}$ is a total function. For \xti{any} $x$
    in the domain $\N$, $f$ can add $3$ to it, and give you a result.
\end{definition}

\begin{definition}[Partial function]
    If a function $f$ is \xtb{not} able to operate on every value in its domain,
    $f$ is a \xti{partial function}.

    The standard example is what's called the \term{reciprocal function}:

    \begin{equation}
        \eva{r}{x} = \frac{1}{x}
    \end{equation}

    Note that $r$ does \xtb{not} have the type $\N \to \N$. In this case, $r$
    instead has the type $\Z \to \Q$, where

    \begin{itemize}
      \item $\Z$ is the set of integers: $\mset{\dots,-2,-1,0,1,2,\dots}$, and
      \item $\Q$ is the set of all numbers that can be expressed as a fraction
        of two integers.
    \end{itemize}

    For every value in $\Z$, \xti{except} $0$, you can run $r$ on it and get a
    result in $\Q$. Remember, you can't divide by $0$.

    But, because $r$ can't operate on every value in its domain, $r$ is a
    partial function.

    Partial functions are not very fun. They don't provide much in the way of
    interesting insights, or have any deeper meaning. The only purpose they
    serve is to be a pain in the ass. For that reason, mathematicians do
    everything they can to avoid them.

    \xtb{Remember:} partial functions are evil!
\end{definition}

\begin{definition}[Codomain]
    Mathematicians have a fancy term for the input type to a function. It
    therefore stands to reason that mathematicians would have an even fancier
    and more pretentious term for the output type: \term{codomain}.

    In general, in math, if you know what \term{something} is, the
    \term{cosomething} is just the dual of \term{something}. In this case, the
    \term{domain} is the input type of the function, and the \term{codomain} is
    the output type of the function.

    If $f : A \to B$, then 

    \begin{equation}
        \codom{f} = B
    \end{equation}

    \xtb{What if $f$ has a curried type, so instead of $A \to B$, it's
      $A \to B \to C \to D \to \dots$?}

    Good question! Remember that curried types are right-associative, so

    \begin{equation}
        A \to B \to C \to D \to \dots
    \end{equation}

    is really just the lazy man's way of writing

    \begin{equation}
        A \to \parens{B \to \parens{C \to \parens{D \to \dots}}}
    \end{equation}

    Therefore, if 

    \begin{equation}
        f : A \to \parens{B \to \parens{C \to \parens{D \to \dots}}}
    \end{equation}

    then

    \begin{equation}
        \codom{f} = B \to \parens{C \to \parens{D \to \dots}}
    \end{equation}
\end{definition}

\sss{Same function, different [co]domain}

This touches upon a cool thing. Let's say we have some function:

\begin{equation}
    f : A \to B \to C \to D
\end{equation}

The type of $f$ is written in curried form. Thus,

\begin{eqnarray}
    \dom{f} & = & A \\
    \codom{f} & = & B \to C \to D \\
\end{eqnarray}

If we write $f$'s type in uncurried form:

\begin{equation}
    f : \parens{A,B,C} \to D
\end{equation}

Then

\begin{eqnarray}
    \dom{f} & = & \parens{A,B,C} \\
    \codom{f} & = & D \\
\end{eqnarray}

So, we have two identical functions with a different domain and codomain. That's
pretty cool.

Okay, I guess it's not \xti{all} that cool. I realized it while I was writing
this out though, so I thought it was worth mentioning.

\begin{definition}[Range]
    Remember when we were talking about domains and total functions? It was just
    the last page. Essentially:

    \begin{itemize}
      \item The \term{domain} is the possible inputs to the function.
      \item A \term{total function} is a function able to operate on all its
        potential inputs.
      \item The \term{codomain} is the possible outputs of the function.
      \item The \xtb{\term{range}} is the actual set of outputs of the function.
    \end{itemize}
\end{definition}

\begin{definition}[Surjection]
    If a functions range is the same as its codomain, then the function is
    \term{surjective}.

    That is, if there's a function $f : A \to B$, then $f$ is surjective if
    there aren't any values of $B$ that can't be ``reached'' by $f$.
\end{definition}

\begin{example}
    Let's say we have some function $f$.

    \begin{description}
      \item[$f$'s domain] The natural numbers less than $10$. That is,

        \begin{equation}
            \dom{f} = \scomp{x : \N}{x < 10}
        \end{equation}

      \item[$f$'s codomain] The natural numbers whose value is at least 5 but
        less than 15. That is,

        \begin{equation}
            \codom{f} = \scomp{x : \N}{5 \leq x < 15}
        \end{equation}
    \end{description}

    Question: how do you write $f$'s signature?

    Answer: like this:

    \begin{equation}
        f : \scomp{x : \N}{x < 10} \to \scomp{x : \N}{5 \leq x < 15}
    \end{equation}

    Okay, let's find a function that fits that signature! Can you make an
    educated guess? I'll let you guess, and put in a page break.

    \newpage

    Here's my answer:

    \begin{equation}
        \eva{f}{x} = x + 5
    \end{equation}

    Let's make a little table, shall we?

    \begin{tabu}{c|c}
      Input value & Output value \\
      \hline \\
      0 & 5 \\
      1 & 6 \\
      2 & 7 \\
      3 & 8 \\
      \dots & \dots \\
      9 & 14 \\
    \end{tabu}

    It just adds 5 to all the input values.

    Let's ask ourselves some questions:

    \begin{enumerate}
      \item Is $f$ total?
      \item Is $f$ surjective?
    \end{enumerate}

    What do those terms mean again?

    \begin{description}
      \item[Total] means that $f$ has an answer for every value in its
        domain. $f$ does, so yay!
      \item[Surjective] means that $f$ reaches every value in its codomain. $f$
        does!
    \end{description}

    Yay, this isn't hard! It's not very interesting yet, but nonetheless it
    isn't hard!
\end{example}

That last function had another property which proves to be important:
injection. That is, for some value in the range, there is exactly one value
in the domain that reaches it.

For instance, $6$ can only be reached by $\eva{f}{1}$. There isn't another value
$x$ such that $\eva{f}{x} = 6$. The \xti{only} value for which that holds is
$1$.

The same property holds for the rest of the range: for every value $y$ in the
range, there is exactly one value $x$ such that $\eva{f}{x} = y$

\begin{definition}[Injection]
    If $f$'s domain is in one-to-one correspondence with
    its range, then $f$ is an injection.

    More formally,

    \begin{textmath}
        Let $f : A \to B$, and $C$ be the range of $f$ (i.e. the subset of the
        codomain reachable by $f$).

        If, for all $y : C$, there is exactly one value $x : A$ such that
        $\eva{f}{x} = y$, then $f$ is an \term{injection}.
    \end{textmath}
\end{definition}

\begin{definition}[Bijection]
    If $f$ is both an injection and a surjection, then $f$ is a
    \term{bijection}.

    More formally,

    \begin{textmath}
        Let $f : A \to B$.

        If, for all $y : B$, there is exactly one value $x : A$ such that
        $\eva{f}{x} = y$, then $f$ is a \term{bijection}.
    \end{textmath}
\end{definition}

\begin{example}
    Let's come up with a total function that is surjective but not injective.

    This is a bit of a challenge! Basically, a function has to go over the same
    value twice in order to not be injective, but it has to go over every value
    in the codomain!

    Here's my version:

    \begin{eqnarray}
        f : \N \to \scomp{x : \N}{x < 10} \\
        \eva{f}{x} = x \mod{10} \\
    \end{eqnarray}

    Okay, what the hell is mod?

    ``mod'' is short for ``modulo''. $x \mod y$ is the remainder of
    non-fractionally dividing $x$ by $y$. For instance, $\frac{12}{10}$ is $1$
    with a remainder of $2$. So $12 \mod{10} = 2$.

    You should notice that $14 \mod{10}$ is $4$, but so is $4 \mod{10}$. And for
    that matter, so is $114 \mod{10}$. There are multiple values that can reach
    $4$ via $f$, so $f$ is \xtb{not} injective.

    However, every value in $f$'s codomain is reached by $f$, and therefore $f$
    is surjective!

    A much easier function would be something like this:\footnote{The pipe in
      the middle just means ``such that'', just as it does in the comprehension
      notation.}

    \begin{eqnarray}
        f : \N \to \N \\
        \eva{f}{0} = 44 \\
        \eva{f}{x} \mid x > 0 = x + 4 \\
    \end{eqnarray}

    However, I thought of the modulo example first, and it seems a bit more
    likely to happen in the real world.
\end{example}

\begin{ExcList}
    \Exercise{Come up with a partial function.}
    \Answer{Many many correct answers here. The one I thought of is:

      \begin{eqnarray}
          f : \N \to \Q \\
          \eva{f}{x} = \frac{2}{x} \\
      \end{eqnarray}
    }

    \Exercise{Come up with a total function that is injective but not
      surjective.}

    \Answer{Again, there are a lot of correct answers here, but the one I came
      up with is:

      \begin{eqnarray}
          f : \N \to \N \\
          \eva{f}{x} = 2x \\
      \end{eqnarray}

      Remember, $xy$ is shorthand for multiplying $x$ and $y$.
    }

    \Exercise{Come up with a total bijective function.}
    \Answer{Here's mine:

      \begin{eqnarray}
          \eva{f}{x} = x
      \end{eqnarray}

      It's so simple, it should be cheating. But it's not.

      \begin{quotation}
          When the book author does it that means that it is not cheating.

          --- Richard M. Nixon
      \end{quotation} 
    }
\end{ExcList}