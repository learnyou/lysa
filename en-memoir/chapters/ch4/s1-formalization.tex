\s{Formalization of set theory}

This section outlines, but does not dive into the formalization of set theory.

\ss{Sets of sets}

I should mention the empty set, called $\nil$. The set $\nil$ is a set with no
elements. It's not all that interesting.

We can also have sets containing other sets:

\begin{equation}
    \mset{\nil}
\end{equation}

As demonstrated by $\N$, we can have infinite sets. We can also have infinite
sets that contain themselves:

\begin{alignmath}{rl}
    \{ & \nil \\
    ,  & \mset{\nil} \\
    ,  & \begin{tabu}{rl}
             \{ & \nil \\
             ,  & \mset{\nil} \\
             \} & \\
         \end{tabu} \\
    ,  & \begin{tabu}{rl}
             \{ & \nil \\
             ,  & \mset{\nil} \\
             ,  & \mset{\nil, \mset{\nil}} \\
             \} & \\
         \end{tabu} \\
    ,  & \begin{tabu}{rl}
             \{ & \nil \\
             ,  & \mset{\nil} \\
             ,  & \mset{\nil, \mset{\nil}} \\
             ,  & \begin{tabu}{rl}
                      \{ & \nil \\
                      ,  & \mset{\nil} \\
                      ,  & \mset{\nil, \mset{\nil}} \\
                      \} & \\
                  \end{tabu} \\
             \} & \\
         \end{tabu} \\
    ,  & \dots \\
    \} &  \\
\end{alignmath}

Take a minute to try to wrap your head around that set. Then, take an hour to
try to realize that the set contains itself. It's not something I can really
explain, you sort of have to realize it on your own.

\ss{Set comprehensions}

Remember the ``type bracket'' notation?

\begin{equation}
    \scomp{x \in N}{x \ge 4} = \mset{4,5,6,7,\dots}
\end{equation}

With sets, that notation is called \term{set comprehension} or \term{set
  builder} notation.

There's a slight little problem with that notation. Let's consider the set of
all sets, and call it $S$.

We can sort of do this wacky thing. Let's consider all the sets that contain
themselves:

\begin{equation}
    \scomp{x \in S}{S \in S}
\end{equation}

That looks weird, but it isn't intrinsically problematic. Here is where the
issue turns up:

\begin{equation}
    T = \scomp{x \in S}{S \notin S}
\end{equation}

So, $T$ the set of all sets that don't contain themselves. It is slightly wacky,
but mostly straightforward.

Is $T \in T$?

Well, that's certainly a weird question. $T$ is the set of sets that don't
contain themselves, so if $T$ contained itself, it would be a
contradiction. Therefore, $T \notin T$.

Now we have that $T \notin T$, and therefore that $T \in T$.

This is a problem, where we can conclude two contradictory things.

\pg{So, what's the solution?} Whoa, there kid, don't get ahead of yourself.

This paradox was first proposed by a philosopher slash mathematician named
Bertrand Russell. It's therefore called ``Russell's paradox''. It highlighted
the need for a formal axiomization of set theory. Previously, mathematicians had
just sort of haphazardly discussed the notion of sets without formally defining
the concept.

\begin{figure}[ht]
    \centering
    \inclgraph{images/russell.png}
    \caption{Bertrand Russell. Source: \cite{russellpic}.}
\end{figure}

There are a number of axiomizations of set theory. The most popular is called
Zermelo-Fraenkel, or simply ZF:

\begin{description}
  \item[1. Axiom of extensionality] If two sets have the same elements, they are
    equal.
  \item[2. Axiom of power set] For every set $X$, there is a ``power set'',
    denoted $2^X$.
\end{description}
