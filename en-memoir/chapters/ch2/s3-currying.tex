\section{Currying and Uncurrying!}

Those exercises were fun, weren't they? I bet not. If you had a lot of trouble
with them, don't worry. They're purposefully difficult.

My experience is that most textbooks have a ton of pedantic, tedious
exercises. I try to not do that. The exercises I gave you are difficult, but the
good kind of difficult.

Anyway, let's move on, hopefully to something a little bit less boring.

Remember $\Z$, the integers? If you don't:

\begin{equation}
    \Z = \mset{\dots,-5,-4,-3,-2,-1,0,1,2,3,4,5,\dots} \\
\end{equation}

Consider this function:

\begin{eqnarray}
    f : \parens{\Z, \Z, \Z} \to \Z \\
    \eva{f}{a,b,c} = a + b + c
\end{eqnarray}

What's $\eva{f}{1,2,4}$? Well,

\begin{eqnarray}
    \eva{f}{1,2,4} & = & 1 + 2 + 4 \\
                   & = & 7 \\
\end{eqnarray}

What happens when you only apply one value to $f$?

\begin{eqnarray}
    \eva{f}{18,b,c} & = & 18 + b + c \\
\end{eqnarray}

I've sort of transformed $f$: instead of being a function that takes three
integers that adds them together, it's now a function that takes one integer,
and returns back a function that takes two integers, and adds them to the first
integer.

That word jumble is confusing. Let me make it symbolic for you:

Previously, $f$ was like this:

\begin{equation}
    f : \parens{\Z, \Z, \Z} \to \Z
\end{equation}

However, if you allow only one value to be sent to $f$, then you can think of
$f$ more like this:

\begin{equation}
    f : \Z \to \parens{\parens{\Z, \Z} \to \Z}
\end{equation}

Let's rewrite $f$ with this new definition in mind:

\begin{equation}
    \eva{f}{a} = \fn{y,z}{a + y + z}
\end{equation}

This might be a little clearer if I used the non-named function notation.

\begin{equation}
    f = \fn{a}{\parens{\fn{y,z}{a + y + z}}}
\end{equation}

Well, let's just consider that second inside function.

\begin{equation}
    \fn{y,z}{a + y + z}
\end{equation}

(This assumes that $a$ was already defined elsewhere, which it would have been
if we put it back where it goes).

Let's look at the type of the inside function

\begin{equation}
    \fn{y,z}{a + y + z} : \mvec{\Z,\Z} \to \Z
\end{equation}

What if you only send the first value to the inside function?

\begin{equation}
    \fn{14,z}{a + 14 + z} : \Z \to \Z
\end{equation}

Well, again, we just transformed the inside function from

\begin{equation}
   \mvec{\Z,\Z} \to \Z
\end{equation}

to something like

\begin{equation}
   \Z \to \parens{\Z \to \Z}
\end{equation}

Well, now, let's sort of consider the original function, $f$:

$f$'s type is now

\begin{equation}
   f : \Z \to \parens{\Z \to \parens{\Z \to \Z}}
\end{equation}

And $f$ is more properly written like this:

\begin{equation}
    f = \fn{a}{\parens{\fn{b}{\parens{\fn{c}{a + b + c}}}}}
\end{equation}

This is called \term{Currying}.

\begin{definition}[Currying]
    \label{def:currying}
    Taking something with type

    \begin{equation}
        \mvec{a,b,c,d,e,\dots} \to \beta
    \end{equation}

    and turning it into something with the type

    \begin{equation}
        a \to \parens{b \to \parens{c \to \dots}}
    \end{equation}
\end{definition}

\term{Uncurrying} is doing exactly the opposite thing

\begin{definition}[Uncurrying]
    \label{def:uncurrying}
    Taking something with type

    \begin{equation}
        a \to \parens{b \to \parens{c \to \dots}}
    \end{equation}

    and turning it into something with the type

    \begin{equation}
        \mvec{a,b,c,d,e,\dots} \to \beta
    \end{equation}
\end{definition}

I'll give some examples of Currying and Uncurrying here, as well as some
exercises. After that, though, I'm going to silently curry \& uncurry types
without explicitly saying that I'm doing so. I do it so often that it's too much
effort to say so every time. You'll get used to it, I promise.

One more thing. I don't like typing out all of the parentheses in this:

\begin{equation}
    a \to \parens{b \to \parens{c \to \dots}}
\end{equation}

Especially when most of the time, they group to the right like that. Instead,
I'm going to write:

\begin{equation}
    a \to b \to c \to \dots
\end{equation}

\ss{Examples}

\begin{example}
    The curried form of:

    \begin{equation}
        \mvec{\Z,\N} \to \Q
    \end{equation}

    is:

    \begin{equation}
        \Z \to \N \to \Q
    \end{equation}
\end{example}

\begin{example}
    The curried form of

    \begin{equation}
        \mvec{a, a \to b} \to b
    \end{equation}

    is:

    \begin{equation}
        a \to \parens{a \to b} \to b
    \end{equation}

    A function that does this is:

    \begin{eqnarray}
        f : a \to \parens{a \to b} \to b \\
        \eva{f}{x,g} = \eva{g}{x}
    \end{eqnarray}
\end{example}


\begin{example}
    \label{exp:flip}
    The curried form of

    \begin{equation}
        \parens{\mvec{a, b} \to c} \to \parens{\mvec{b,a} \to c}
    \end{equation}

    is:

    \begin{equation}
        \parens{a \to b \to c} \to b \to a \to c
    \end{equation}

    The function with this exact type is $\flip$.

    \begin{eqnarray}
        \flip & : & \parens{a \to b \to c} \to b \to a \to c \\
        \eva{\flip}{f,x,y} & = & \eva{f}{y,x} \\
    \end{eqnarray}
\end{example}

% TODO: Add more exercises
\begin{ExcList}
    \Exercise{Write out the uncurried and curried type of the $+$ function over
      natural numbers.}
    \Answer{Curried:

        \begin{equation}
            \N \to \N \to \N
        \end{equation}

        Uncurried:

        \begin{equation}
            \mvec{\N, \N} \to \N
        \end{equation}
    }

    \Exercise{There is a very important function, called ``compose''. The symbol
        for compose is a circle: $\circ$. Here is compose's type:

        \begin{equation}
            \circ : \parens{b \to c} \to \parens{a \to b} \to a \to c
        \end{equation}

        It is usually used in infix form:

        \begin{eqnarray}
            f : a \to b \\
            g : b \to c \\
            g \circ f : a \to c \\
        \end{eqnarray}

        Write the $\circ$ function. 
    }
    \Answer{$\eva{f \circ g}{x} = \eva{f}{\eva{g}{x}}$}
\end{ExcList}