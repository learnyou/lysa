\section{Functions}

Okay, so what's a function?

A function is a mathematical construct that takes some input, does something to
it, and sends back output. Essentially, instead of being some static value that
isn't very interesting, a function takes something as input, does something to
it, and spits back some output.

Let's make a function that takes some number, and adds 3 to it.

\begin{equation}
    \fn{x}{x + 3}
\end{equation}

\xtb{What is that notation?}

The first character is the Greek letter lambda: $\lambda$. It serves the same
function in Greek as the 'L' does in English. The $\lambda$ serves to indicate
that we are creating a function, not just an ordinary value.\footnote{Although a
  lot of people will use $\lambda$ as an ordinary variable name, which gets very
  confusing. Thankfully, I haven't seen anyone use \xti{both} the $\lambda$ for
  functions and for a variable name.}

Immediately following the $\lambda$ is a list of inputs to the function. In this
case, there is only one, which is called $x$. If there was more than one input,
they would be separated by commas. After the list of variables, there's a
weirdly shaped arrow: $\mapsto$. It should be read as ``maps to''.

The portion after the arrow indicates what the function does to the inputs. This
is called, uninterestingly enough, the \term{output} of the function.

That's all a function is: input and output. You've likely forgotten what the
function is, so I'll write it again:

\begin{equation}
    \fn{x}{x + 3}
\end{equation}

This function takes some number $x$ as input, and the output is $x + 3$. This is
not a very interesting function, I know.

Let's give this function a name

\begin{equation}
    f = \fn{x}{x + 3}
\end{equation}

The much more common notation is this:

\begin{equation}
    \eva{f}{x} = x + 3
\end{equation}

That is to be read ``$f$ of $x$ equals $x$ plus $3$''.

What's the type of $f$?

\begin{alignedmath}
    f : \N \to \N \\
    \eva{f}{x} = x + 3
\end{alignedmath}

Remember that $\N = \mset{0,1,2,3,4,5,6,\dots}$.

I think the $\N \to \N$ part is pretty self-explanatory: $f$ takes something of
type $\N$, and spits back out something of type $\N$.

So, what happens when we feed $80$ to $f$?

\begin{rclmath}
    \eva{f}{80} & = & 80 + 3 \\
                & = & 83 \\
\end{rclmath}

Yay! This math stuff isn't hard!

\ss{Comprehensions}

Think about this for a second: $f$ takes a number and adds 3 to it. What's the
lowest value in $\N$? Well, it's $0$. If you take $0$, and add $3$ to it, you
get $3$. You can't add $3$ to a natural number and get something less than
$3$.

You would therefore be more correct in saying that $f$'s output type is the set
of numbers greater than (or equal to) $3$.

\begin{rclmath}
    f & : & \N \to \comprehension{z : \N}{z \ge 3} \\
    \eva{f}{x} & = & x + 3 \\
\end{rclmath}

\xtb{What is that curly brace nonsense?}

It's called a \term{comprehension}.

\answergraph{images/comprehensions.png}

Read it in two parts: before the pipe, and after the pipe.

\begin{itemize}
  \item Before the pipe describes what every item of the type looks like. In
    this case, it's just a number $z$.

  \item After the pipe has a list of conditions that must hold true. In this
    case, $z$ must be greater than or equal to $3$.
\end{itemize}

\answergraph{images/comprehensions3.png}

Let's look through some more comprehensions.

\begin{example}
    The type of all natural numbers greater than $60$, but less than $70$:

    \begin{equation}
        \comprehension{x : \N}{60 < x < 70}
    \end{equation}

    Alternatively, you can use the logical-and operator:

    \begin{equation}
        \comprehension{x : \N}{\parens{60 < x} \text{ and } \parens{x < 70}}
    \end{equation}

    That should be read as ``the type of all numbers $x$ such that $60$ is less
    than $x$ and $x$ is less than $70$''.
\end{example}

\begin{example}
    A function that takes some number $x$, and constructs the type of all
    natural numbers less than $x$

    \begin{equation}
        \fn{x}{\comprehension{y : \N}{y < x}}
    \end{equation}
\end{example}

\begin{example}
    The type of all natural numbers that are a multiple of two:

    \begin{equation}
        \comprehension{2x : \N}{x : \N}
    \end{equation}

    This is a bit odd. In this case, it's easier to read the stuff after the
    pipe first:

    \begin{equation}
        x : \N
    \end{equation}

    So, for every number $x$, we're going to multiply it by $2$. Therefore,
    there won't be any numbers in the result that aren't multiples of $2$. Yay!
\end{example}

\begin{example}
   The Fibonacci numbers are the numbers 

   \begin{equation}
       F = \mvec{1,1,2,3,5,8,13,21,34,55,89, \dots}
   \end{equation}

   \xtb{Why did you use round parentheses instead of curly braces?}

   The Fibonacci numbers have repetition, and the order in which you list them
   does matter. To represent this, you use round parentheses when listing them
   instead of curly braces.\footnote{The mathematical difference is that the
     Fibonacci numbers are a \term{list}, while the other types we've
     encountered are all \term{sets}. I'll get to these two in depth later on in
     the book.}

   The $n$th Fibonacci number is the last two Fibonacci numbers added
   together. $89 = 55 + 34$. $55 = 34 + 21$. Et cetera.

   This is a function that gives the $n$th Fibonacci number:

   \begin{rclmath}
       \eva{f}{0} &=& 1 \\
       \eva{f}{1} &=& 1 \\
       \eva{f}{n} &=& \eva{f}{n - 1} + \eva{f}{n - 2}
   \end{rclmath}

   When indexing items, you always start at $0$. It's just a thing.

   This is an example of \term{recursion}: the definition of $f$ uses itself in
   its definition.
\end{example}

\begin{example}
    Natural numbers greater than $1$ can be divided into two categories:

    \begin{description}
      \item[Composite numbers] can be divided by another natural number to get a
        third natural number. For instance, $15$ can be divided by $3$ to get
        $5$.

        There are always two \term{trivial divisors}. $1$ can trivially divide
        every number. Likewise, every number can trivially divide itself. These
        trivial divisors occur in every single number and thus are not very
        interesting.

      \item[Prime numbers] have only trivial divisors. For instance, there is no
        non-trivial divisor of $7$.
    \end{description}

    \xtb{Why do you exclude $1$?}

    Well, because $1$ has only trivial divisors.

    \xtb{Okay, but what about $2$?}

    Well\dots hell, I dunno\dots

    Oh, oh, I remember! There's this cool thing called the Sieve of
    Eratosthenes, I'll show it to you in a little while.\footnote{TODO: show
      this to reader.} Anyway, the sieve makes sense if $2$ is prime but $1$
    isn't.

    Let's construct the type of composite numbers:

    \begin{equation}
        \label{eq:composite}
        \scomp{x \cdot y : \N}{
          \parens{x : \N}
          \text{ and }
          \parens{y : \N}
          \text{ and }
          \parens{x > 1}
          \text{ and }
          \parens{y > 1}
        }
    \end{equation}

    Let's think about this:

    If you hold $x = 2$ (the minimum value of $x$), let's see what the output
    values are:

    \begin{eqnarray}{cl}
       & \scomp{2 \cdot y : \N}{
           \parens{2 : \N}
           \text{ and }
           \parens{y : \N}
           \text{ and }
           \parens{2 > 1}
           \text{ and }
           \parens{y > 1}
         } \\
       = & \scomp{2 \cdot y : \N}{
             \parens{y : \N}
             \text{ and }
             \parens{y > 1}
           } \\
       = & \mset{2 \cdot 2, 2 \cdot 3, 2 \cdot 4, 2 \cdot 5, 2 \cdot 6, \dots} \\
       = & \mset{4, 6, 8, 10, 12,  \dots} \\
    \end{eqnarray}

    That is, all of the multiples of $2$ greater than $2$. Again, $2$ trivially
    divides $2$, so I excluded it (this is why $x$ has to be greater than $1$,
    because otherwise $2$ would be in the list of multiples of $2$).

    Now, let's hold $x = 3$

    \begin{equation}
        \mset{6, 9, 12, 15, 18, 21, 24, \dots}
    \end{equation}

    With $x = 4$

    \begin{equation}
        \mset{4, 8, 12, 16, 20, 24, 28, \dots}
    \end{equation}

    You get the point. If you follow through with this (infinitely), you would
    eventually list every single composite number.

    \xtb{What about $2$? Isn't $2$ a composite number?}

    Nope! $2$ can only be divided by $1$ or $2$, which makes it prime.

    \xtb{What about dividing by $0$?}

    You aren't allowed to divide by $0$.

    \xtb{Why}?

    Because allowing people to divide by $0$ would poke holes into our logic.

    \xtb{But if you take a cookie, and break it into $0$ parts, then you get
      $0$, right?}

    Jesus, with the questions\dots Will you give it a rest? I'll explain later,
    I promise.
\end{example}

I'm going to give you some exercises. Please do them. They'll really help you
with the material.

\begin{ExerciseList}
    \Exercise{Using that weird type bracket notation, construct the type of all
      numbers greater than $80$, and less than $200$.}
    \Answer{\(\comprehension{x : \N}{80 < x < 200 }\)}

    \Exercise{Construct the type of all numbers less than or equal to $30$, and
      greater than $10$.}
    \Answer{\(\comprehension{x : \N}{10 < x \le 30 }\)}

    \Exercise{
      \label{xstotype}

      Construct a function that takes two numbers $x$ and $y$, and uses
      them to construct the type of numbers greater than $x$ but less than $y$.

      Hint: to have multiple values sent to a function, just add commas: $\fn{x,
      y}{x + y}$}
    \Answer{\(\fn{x, y}{\comprehension{a : \N}{x < a < y }}\)}

    \Exercise{Figure out the type of the function you constructed in
      \cref{xstotype}.}
    \Answer{$\mlist{\N, \N} \to \Type$}

    \Exercise{The factorial of a number $n$, written

      \begin{equation}
          n!
      \end{equation}

      Is $n$ multiplied by all of the numbers less than $n$. Write a recursive
      function to get the factorial of any natural number.

      Hint: $0! = 1$.
    }
    \Answer{
      Here's what I went with:

      \begin{eqnarray}
          0! = 1 \\
          n! = n \cdot \parens{\parens{n - 1}!}
      \end{eqnarray}
    }
\end{ExerciseList}