\ch{Sets}

So, I lied to you a little bit. Mathematicians don't usually use types. Some do,
but most don't. \term{Sets} are almost the same thing as types, and provide the
same abstraction capabilities. People tend to confuse sets and lists, so I
started with types, because the distinction is much clearer.

Informally, a set is a bunch of objects, with no particular order or notion of
duplication. That sounds like nonsense. In less nonsensical form:

\begin{equation}
    \mset{1,2,3} \text{ is a set.}
\end{equation}

Duplicate elements are not counted separately, so

\begin{equation}
    \mset{1,1,2,3}
\end{equation}

is the same set as $\mset{1,2,3}$. The order in which elements appear does not
matter, so 

\begin{equation}
    \mset{3,1,2,1,2}
\end{equation}

is also the same set.

Well, that's half true. Some sets are ordered, but every set does not have an
ordering.

Instead of the colon, the symbol $\in$ is used to indicate that some item is in
a set.

Remember the natural numbers, $\N$? If not,

\begin{equation}
    \N = \mset{0,1,2,3,4,5,\dots}
\end{equation}

All of arithmetic can be formalized with these 5 axioms. \cite{taylor-analysis, landau-analysis}

\begin{enumerate}
  \item $0$ is a natural number.
  \item For every natural number $n \in \N$, there is an immediate successor
    $\eva S n \in \N$.
  \item If $\eva S n = \eva S m$, then $m = n$. (In our previous terms, $S$ is
    an injection).
  \item There is no number $n \in \N$ such that $\eva S n = 0$
  \item This is a complicated expression:

    \begin{itemize}
      \item If there is a set $K$ such that
      \item $0 \in K$,
      \item the fact that $n \in K$ implies that $\eva S n \in K$,
      \item then $K$ contains every natural number.
    \end{itemize}
\end{enumerate}

The formalization of arithmetic from these axioms is incredibly boring, so I
won't cover it. If you really want to know about it, I recommend this book:
\fullcite{landau-analysis}.

The fifth axiom in that list sounds like nonsense, but it is actually incredibly
important. It sets up the basic idea of \term{mathematical induction}. The idea
is this:

\begin{itemize}
  \item Say you have some proposition, and the proposition is dependent on a
    number.
  \item For instance, $0 \le x$ is dependent on whatever $x$ is.
  \item If you know the proposition is true for $0$, and
  \item The fact that the proposition is true for some natural number $n$,
    implies that the proposition is true for $n$'s successor $\eva S n$. That is

    \begin{equation}
        \eva P n \implies \eva{P}{\eva S n}
    \end{equation}
  \item Then the proposition is true for every natural number.
\end{itemize}

It's an incredibly important proof technique.

1. Null set
2. Infinite sets
3. Sets containing other sets.
4. Comprehensions
5. Set equality
6. equivalence relations
7. Unions, intersections, subtractions, complements
8. Power sets
9. Cardinality
10. Russell's paradox
11. ZF
12. Axiom of choice and Zorn's lemma
13. The notion of proper classes

Next section

Different things that sets can do

1. Monoids, semigroups, groups, magmas
2. Rings
3. Fields

Next section

Vector spaces + linear algebra

Next section

Calculus

\s{Formalization of set theory}

This section outlines, but does not dive into the formalization of set theory.

\ss{Sets of sets}

I should mention the empty set, called $\nil$. The set $\nil$ is a set with no
elements. It's not all that interesting.

We can also have sets containing other sets:

\begin{equation}
    \mset{\nil}
\end{equation}

As demonstrated by $\N$, we can have infinite sets. We can also have infinite
sets that contain themselves:

\begin{alignmath}{rl}
    \{ & \nil \\
    ,  & \mset{\nil} \\
    ,  & \begin{tabu}{rl}
             \{ & \nil \\
             ,  & \mset{\nil} \\
             \} & \\
         \end{tabu} \\
    ,  & \begin{tabu}{rl}
             \{ & \nil \\
             ,  & \mset{\nil} \\
             ,  & \mset{\nil, \mset{\nil}} \\
             \} & \\
         \end{tabu} \\
    ,  & \begin{tabu}{rl}
             \{ & \nil \\
             ,  & \mset{\nil} \\
             ,  & \mset{\nil, \mset{\nil}} \\
             ,  & \begin{tabu}{rl}
                      \{ & \nil \\
                      ,  & \mset{\nil} \\
                      ,  & \mset{\nil, \mset{\nil}} \\
                      \} & \\
                  \end{tabu} \\
             \} & \\
         \end{tabu} \\
    ,  & \dots \\
    \} &  \\
\end{alignmath}

Take a minute to try to wrap your head around that set. Then, take an hour to
try to realize that the set contains itself. It's not something I can really
explain, you sort of have to realize it on your own.

\ss{Set comprehensions}

Remember the ``type bracket'' notation?

\begin{equation}
    \scomp{x \in N}{x \ge 4} = \mset{4,5,6,7,\dots}
\end{equation}

With sets, that notation is called \term{set comprehension} or \term{set
  builder} notation.

There's a slight little problem with that notation. Let's consider the set of
all sets, and call it $S$.

We can sort of do this wacky thing. Let's consider all the sets that contain
themselves:

\begin{equation}
    \scomp{x \in S}{S \in S}
\end{equation}

That looks weird, but it isn't intrinsically problematic. Here is where the
issue turns up:

\begin{equation}
    T = \scomp{x \in S}{S \notin S}
\end{equation}

So, $T$ the set of all sets that don't contain themselves. It is slightly wacky,
but mostly straightforward.

Is $T \in T$?

Well, that's certainly a weird question. $T$ is the set of sets that don't
contain themselves, so if $T$ contained itself, it would be a
contradiction. Therefore, $T \notin T$.

Now we have that $T \notin T$, and therefore that $T \in T$.

This is a problem, where we can conclude two contradictory things.

\pg{So, what's the solution?} Whoa, there kid, don't get ahead of yourself.

This paradox was first proposed by a philosopher slash mathematician named
Bertrand Russell. It's therefore called ``Russell's paradox''. It highlighted
the need for a formal axiomization of set theory. Previously, mathematicians had
just sort of haphazardly discussed the notion of sets without formally defining
the concept.

\begin{figure}[ht]
    \centering
    \inclgraph{images/russell.png}
    \caption{Bertrand Russell. Source: \cite{russellpic}.}
\end{figure}

There are a number of axiomizations of set theory. The most popular is called
Zermelo-Fraenkel, or simply ZF:

\begin{description}
  \item[1. Axiom of extensionality] If two sets have the same elements, they are
    equal.
  \item[2. Axiom of power set] For every set $X$, there is a ``power set'',
    denoted $2^X$.
\end{description}

